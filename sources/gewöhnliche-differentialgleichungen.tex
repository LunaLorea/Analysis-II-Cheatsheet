% Tex root = ../cheatsheet.tex
\section{Gewöhnliche Differentialgleichungen (ODE)}
\definition{2.1.0.1}{ODE}
Sei $k\geq1$, $U\subseteq\mathbb R^{k+2}$, $G:U\rightarrow\mathbb R$. Dann ist 
$$G(x,y,y',y'',...,y^{(k)})=0$$ eine ODE $k$-ter Ordnung.\\
\definition{2.1.0.2}{Lösung einer ODE}
Eine Lösung einer ODE der Ordnung $k$ ist eine $k$-mal diffbare Funktion $f:I\rightarrow\mathbb R$
auf einem offenen Intervall $I\subseteq\mathbb R$ mit $$G(x, f(x). f'(x),...,
f^{(k)}(x)=0$$.\\
\definition{2.1.0.3}{Anfangswertproblem}
Sind bei einer ODE zusäzlich noch Anfangsbedinungen gegeben, dh.
$$y(x_0)=y_0,y'(x_0)=y_1,...,y^{(k-1)}(x_0)=y_k$$ mit $x_0,
y_0,...,y_k\in\mathbb R$, dann liegt ein AWP vor.\\
\rmrk{2.1.0.4}{} k wird generell minimal angegeben. \\
$G(x, y)=0$ ist \textbf{keine} ODE.\\
Lässt sich eine ODE als $y^{(k)}=F(x,y,y',...,y^{(k-1)})$ schreiben so nennen
wir diese \textbf{explizit}.\\
Stammfunktionsprobleme sind spezialfälle einer ODE, eg $y'=1/x$.\\
Sind ODEs nicht von $x$ abhängig so nennt man diese \textbf{Autonom}. Eg.
$y''=1/m$\\
\subsection{Einführung}
\proposition{2.1.6}{Existenz- und Eindeutigkeitssatz}
Ein AWP $y'=F(x,y)$, $y(x_0)=y_0$ mit $F\rightarrow\mathbb R$ stetig für
$U\in\mathbb R^2$ offen, $(x_0,y_0)\in U$, hat eine Lösung.\\
Ist $F$ stetig differenzierbar, so gibt es eine \textbf{eindeutige maximale}
Lösing. (Maximal bedeutet, es ist nicht eine Einschränkung einer anderen Lösung
mit grösserem Intervall.)
\subsection{Other Methods}
\definition{2.6.1}{Separation der Variablen}
Für ODEs der Form $y'=a(y)\cdot b(x)$ und $a,b$ stetig.
Für jede Nulstelle $y_0\in\mathbb R$ von $a$ gibt es eine Konstante Lösung
$y(x)=y_0$.\\
Für $a(y)\neq0$: ODE$\iff \frac{y'}{a(y)}=b(x)$\\
$\iff\int \frac{y'}{a(y)}dx=\int b(x)dx+c$ für $c\in\mathbb R$
\begin{enumerate}
  \item Finde Stammfunktion A,B von $\frac{1}{a}, b$\\
  $*$ Kettenregel: $\int\frac{y'}{a(y)}dx = A(y)+c$\\
  $\implies A(y)=B(x) + c$
  \item Falls A eine Umkehrfunktion hat, dann ist $y=A^{-1}(B(x)+x)$
\end{enumerate}


